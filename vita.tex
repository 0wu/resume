%%!TEX encoding = UTF-8 Unicode

\documentclass[11pt,letterpaper,sans]{moderncv}
\moderncvstyle{classic} % CV theme - options include: 'casual' (default), 'classic', 'oldstyle' and '
\moderncvcolor{blue} % CV color - options include: 'blue' (default), 'orange', 'green', 'red', 'purpl

\usepackage[utf8]{inputenc}

\usepackage[margin=1in]{geometry} % Reduce document margins
\setlength{\hintscolumnwidth}{2cm} % Uncomment to change the width of the dates column
%\setlength{\makecvtitlenamewidth}{9cm} % For the 'classic' style, uncomment to adjust the width of t
\graphicspath{{../}{image/}}
%\renewcommand*{\refname}{Selected Publications}

\usepackage{CJK}
\usepackage{xstring}
\usepackage{comment}

\usepackage{bibentry}
\nobibliography*
\usepackage[numbers]{natbib}

%\usepackage[sorting=none]{biblatex}
%\addbibresource{thesis.bib}

\firstname{Tingfan}
\familyname{Wu 吳亭範}
\title{}
\address{1885 EL Paseo St. Apt 519}{Houston, TX 77054 U.S.A}
\phone{336-837-8244}
\email{tingfan@gmail.com}
%\homepage {mplab.ucsd.edu/\textasciitilde ting/}

\def\libsvm{\httplink[\textbf{libsvm}\homepagesymbol]{www.csie.ntu.edu.tw/\textasciitilde cjlin/libsvm}}

\begin{document}

\begin{CJK*}{UTF8}{bkai}
\makecvtitle
\end{CJK*}

\vspace{-10mm}

%\section{Objective}
%Seeking a full-time research software engineer position in self-driving car perception or control team
%Seeking a full-time robotic research / engineer position %in the IHMC DARPA Robotic Challenge Team

\section{Technical Skills}
	\cvitem{programming}{Matlab, Python, Java, C/C++}
	\cvitem{robotics}{humanoid locomotion, articulate body kinematics/dynamics modeling \& identification, reinforcement motor learning,  sensor fusion, optical motion capture,  optimal control (iterative LQG), pneumatic actuators, Robotic Operating System (ROS)}
	\cvitem{computer vision}{texture recognition, activity classification, facial expression analysis, support vector machines}
	\cvitem{fabrication}{Solidworks, 3D printing, laser cutter, CNC, basic PCB,  8051, Arduino}

	
\section{Professional Experience}
	\cventry{2015-present}{Co-Founder \& Chief Scientist}{}{}{UmboCV Inc.}
	{
	}
	
	\cventry{2013-2015}{Postdoc Researcher}{}{P.I Jerry Pratt}{Florida Institute Human Machine Cognition (IHMC)}
	{
		\begin{itemize}
		\item Integrate whole-body force-control and ICP-based walking algorithm onto Boston Dynamics Atlas humanoid. My primary focus was on 
			low-level hydraulic force control. We won second place on DARPA Robotics Challenge (2015) and DARPA Robotic Challenge Trials (2013).
		\item Lead the IHMC effort porting the IHMC control/walking algorithm onto NASA's humanoid robot, Valkyrie. Besides
		project management, I also focused on series-elastic actuator torque control tuning and robot pose sensor processing.
		\item Integrate vision algorithms for drift-free robot pose estimation and automatic robot testing.
		\end{itemize}
	}
	
	\cventry{2013}{Visiting Scholar}{}{P.I Emanuel Todorov}{University of Washington}
	{
		\begin{itemize}
		\item Constructing model predictive controllers for humanoid robot walking and anthropomorphic robot arm manipulation
		\end{itemize}
	}
	
	\cventry{2010-2013}{Research Assistant}{}{P.I. Javier Movellan}{}
	{
	  \begin{itemize}%\parskip=0pt\vspace*{-3pt}
	  \item Developed an automatic sensor \textbf{calibration} algorithm exploiting redundancy among sensors of multiple modalities. On a task calibrating 38 joint-angle potentiometers and 24 motion capture markers on a humanoid robot, the algorithm achieves  7x higher precision than manual calibration
	  \item Designed a \textbf{semi-parametric dynamics model} which outperforms its parametric and non-parametric counterparts. The proposed model  flexibly fits arbitrary observed dynamics  and  generalizes systematically to unobserved situation 
	  \item Created a novel \textbf{reinforcement learning framework} enabling  humanoid robots to learn motor and social skills in the similar manner as we humans do
	  \end{itemize}
	}
	
	\cventry{2007- 2010}{Research Assistant / Software Engineer}{}{P.I. Marian Bartlett / {\httplink[Emotient\homepagesymbol]{emotient.com}}}{}
	{
		\begin{itemize}
		\item Designed novel computer vision algorithms for spatiotemporal Action Unit based facial expression recognition;
		widely used in emotion related research;  being commercialized by a  startup, Emotient
		\end{itemize}
	}
	
	\cventry{2004}{Research Intern}{}{Siemens Corporate Research}{Princeton, NJ}
	{
	  \begin{itemize}
	  \item Developed software for automatic hearing aids shell customization from 3D ear canal model
	  \item Created a high-accuracy colon polyp detector for computer tomography data
	  \end{itemize}
	  }

	\cventry{2002 - 2004}{Undergraduate Research Assistant}{}{P.I. Chih-Jen Lin}{}
	{
		\begin{itemize}
		\item Created an efficient algorithm for probability estimation for  support vector machines, now implemented in \libsvm
		\end{itemize}
	}
\section{Education}

\cventry{2013}{Ph.D. in Computer Science}{University of California, San Diego (UCSD)}{La Jolla, CA}{}
{
	Thesis: Machine Learning Algorithms for Humanoid Robot Modeling and Control \\
	Advisor: Javier Movellan 
}

\cventry{2010}{M.Sc. in Computer Science}{University of California, San Diego (UCSD)}{La Jolla, CA}{}
{
	Research: Computer Vision -- FACS-based Facial Expression Recognition \\
	Advisor: Javier Movellan \& Marian Bartlett | GPA:3.84 
}


\cventry{2004}{B.Sc in Computer Science}{National Taiwan University (NTU)}{Taipei City, Taiwan}{}
{
  	Research: Probabilistic Output for Support Vector Machines (in \libsvm)  \\
	Advisor: Chih-Jen Lin | GPA: 3.91(major)\\
	President's Award (top 5\% in class) in 2000, 2002 and 2003
}

\section{Software}
\cventry{2004}{libsvm-prob}{}{extensions to a popular support vector machine library to provide probabilistic prediction in addition to original binary predictions (1800$^+$ citations)}{}{}
\cventry{2007-2011}{AULearner}{}{a module for Computer Expression Recognition Toolbox(CERT) for Action Unit based expression recognition}{}{}
\cventry{2012-2013}{ROS-Matlab-Bridge}{}{a library enabling writing ROS node in Matlab \httplink[\homepagesymbol(80$^+$ downloads)]{http://code.google.com/p/mplab-ros-pkg/wiki/java_matlab_bridge}}{}{}

% PUBLICATION section, title created by cls file automatically
%this macro matches my name and bold-face it.

\let\originalbibitem\bibitem
\def\bibitem#1#2\par{%
  \noexpandarg
  \originalbibitem{#1}
  \StrSubstitute{#2}{Tingfan Wu}{\textbf{Tingfan Wu}}
  \StrSubstitute{#2}{Ting-Fan Wu}{\textbf{Ting-Fan Wu}}}
%\newcommand{\mypaper}[1]{\httplink[\homepagesymbol]{mplab.ucsd.edu/\textasciitilde ting/pdfs/#1}}
  
 %now the list of articles to be cited
\bibliographystyle{unsrt}
\bibliography{./thesis.bib}
	\nocite{wu2004probability}
	\nocite{bartlett2006insights}
	\nocite{bartlett2008computer}
	\nocite{wu2009learning}
	\nocite{fasel2009infomax}
	\nocite{wang2009hebbian}
	\nocite{whitehill2009whose}
	\nocite{wu2010facial}
	\nocite{wu2011collecting}
	\nocite{wu2011action}
	\nocite{littlewort2011computer}
	\nocite{littlewort2011motion}
	\nocite{wu2012simultaneous}
	\nocite{wu2012semi}
	\nocite{wu2012multilayer}
	\nocite{ruvolo2012control}
	\nocite{sikka2012exploring}
	\nocite{long2012learning}
	\nocite{wu2013stac}
	\nocite{tassa2013modeling}
	\nocite{johnson2015team}
	\nocite{wiedebach2016walking}
	\nocite{koolen2016design}
	%\nocite{wu2013dna}
%\printbibliography

\section{Presentations}
	\cventry{2011}{``Cross Database Action Unit Recognition Transfer''}{Facial Expression Recognition Analysis Challenge 2011}{Santa Barbara, CA}{(Second Place Winner Talk)}{}{}
	\cventry{2009}{``A Practical Guide to Support Vector Machines''}{a general introduction talk to SVMs presented multiple times in NSF iSLC meeting 2008, UCSD TDLC bootcamp 2009, and  Telluride Neuromorphic  Workshop 2009}{}{}{}
	\cventry{2006}{``Cafeteria Vision -- an automatic dish identification and amount measurement system''}{UCSD All Graduate Student Symposium}{}{}{} %cited by US patent 8229204 B2
	\cventry{2006}{``Ranking by Stealing Human Cycles''}{Workshop in Machine Learning Summer School 2006}{Taipei, Taiwan}{}{}
	\cventry{2000}{``The Colorful Liquid Crystal Filter''}{International Science and Engineering Fair 2000}{Detroit, IL}{(Fourth Award, ranked 7/157)}{}
	
\section{Academic Service}
		\cventry{2011-2013}{Reviewer}{International Conference on Robotics and Automation}{}{}{}
		\cventry{2012}{Communication Chair}{IEEE International Conference on Development and  Learning}{}{}{}
		\cventry{2008}{Student Committee}{UCSD CSE Ph.D Admission Committee}{}{}{}
		\cventry{2006-2007}{Reviewer}{IEEE Transactions on Knowledge and Data Engineering}{}{}{}
		\cventry{2007}{Reviewer}{IEEE Transactions on Neural Networks}{}{}{}
 
\section{Other Experiences}
	\cventry{2004 - 2006}{Information System Officer}{}{R.O.C Air Force Academy} {Kaohsiung, Taiwan}
	{
 		\begin{itemize}\parskip=0pt\vspace*{-3pt}
 			\item Lead a team of 12 people delivering 24hr services to base-wide intranet and 1500+ PCs
			\item Developed a network anomaly detector using support vector machines on netflow data
			\item Trained soldiers without technical experience to diagnose computer problems by customized SOPs and statistical issue tracking systems;
			sped up the throughput by an order of magnitude
 		\end{itemize}
	}
\section{Additional Skills}
 	\cvitem{languages}{English (proficient) Mandarin Chinese (native), Taiwanese (native)}

\end{document}

